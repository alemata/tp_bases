\section{Código}

\subsection{Creación de las Tablas}
\begin{verbatim}
# Tipos de Camaras
CREATE TABLE tipos_de_camara (
    tipo VARCHAR(15) NOT NULL,
    PRIMARY KEY (tipo)
);

# Ciudadanos
CREATE TABLE ciudadanos (
    id INT NOT NULL AUTO_INCREMENT,
    dni VARCHAR(15),
    fecha_nacimiento DATE,
    nombre VARCHAR(63),
    apellido VARCHAR(63),
    PRIMARY KEY (id)
);

#  Camaras
CREATE TABLE camaras (
	id INT NOT NULL AUTO_INCREMENT,
    tipo VARCHAR(15) NOT NULL,
    año INT NOT NULL,
    presidente_id INT,
    PRIMARY KEY (id),
    FOREIGN KEY (tipo) REFERENCES tipos_de_camara(tipo),
    FOREIGN KEY (presidente_id) REFERENCES ciudadanos(id),
    CONSTRAINT unicas_camaras_por_año UNIQUE (tipo, año)
);

CREATE TABLE camaras_senadores (
    id INT NOT NULL,
    PRIMARY KEY (id),
    FOREIGN KEY (id) REFERENCES camaras(id) 
);

CREATE TABLE camaras_diputados (
    id INT NOT NULL,
    PRIMARY KEY (id),
    FOREIGN KEY (id) REFERENCES camaras(id) 
);



CREATE TABLE diputados (
    id INT NOT NULL,	
    PRIMARY KEY (id),
    FOREIGN KEY (id) REFERENCES ciudadanos(id) 
);

CREATE TABLE empleados (
    id INT NOT NULL,
    año INT NOT NULL,	
    PRIMARY KEY (id),
    FOREIGN KEY (id) REFERENCES ciudadanos(id) 
);

# Provincia
CREATE TABLE provincias (
    id INT NOT NULL AUTO_INCREMENT,
    nombre VARCHAR(31),
    poblacion INT,
    PRIMARY KEY (id)
);


# Ternaria entre camara, ciudadano y provincia
CREATE TABLE representaciones (
    ciudadano_id INT NOT NULL,
    camara_id INT NOT NULL,
    provincia_id INT NOT NULL,
    PRIMARY KEY (ciudadano_id, camara_id, provincia_id),
    FOREIGN KEY (ciudadano_id) REFERENCES ciudadanos(id),
    FOREIGN KEY (camara_id) REFERENCES camaras(id),
    FOREIGN KEY (provincia_id) REFERENCES provincias(id),
    CONSTRAINT representa_una_sola_provincia UNIQUE (ciudadano_id, camara_id)
);

# Bloque politico
CREATE TABLE bloques_politicos (
    id INT NOT NULL AUTO_INCREMENT,
    nombre VARCHAR(127),
    PRIMARY KEY (id)
);

# Bloque politico con ciudadadnos (presidentes)
CREATE TABLE bloques_politicos_ciudadanos_presidentes (
    ciudadano_id INT NOT NULL,
    bloque_politico_id INT NOT NULL,
    año INT,
    PRIMARY KEY (ciudadano_id, bloque_politico_id, año),
    FOREIGN KEY (ciudadano_id) REFERENCES ciudadanos(id),
    FOREIGN KEY (bloque_politico_id) REFERENCES bloques_politicos(id) 
);

# Bloque politico con ciudadadnos (integrantes)
CREATE TABLE bloques_politicos_ciudadanos_inegrantes (
    ciudadano_id INT NOT NULL,
    bloque_politico_id INT NOT NULL,
    año INT,
    PRIMARY KEY (ciudadano_id, bloque_politico_id, año),
    FOREIGN KEY (ciudadano_id) REFERENCES ciudadanos(id),
    FOREIGN KEY (bloque_politico_id) REFERENCES bloques_politicos(id) 
);

# Partido político
CREATE TABLE partidos_politicos (
    id INT NOT NULL AUTO_INCREMENT,
    nombre VARCHAR(127),
    PRIMARY KEY (id)
);

# Partido politico con ciudadadnos
CREATE TABLE partidos_politicos_ciudadanos (
    ciudadano_id INT NOT NULL,
    partido_politico_id INT NOT NULL,
    año INT,
    PRIMARY KEY (ciudadano_id, partido_politico_id, año),
    FOREIGN KEY (ciudadano_id) REFERENCES ciudadanos(id),
    FOREIGN KEY (partido_politico_id) REFERENCES partidos_politicos(id) 
);

# Tipos de Camaras
CREATE TABLE tipos_de_sesion (
    tipo VARCHAR(15) NOT NULL,
    PRIMARY KEY (tipo)
);

# Sesion
CREATE TABLE sesiones (
    id INT NOT NULL AUTO_INCREMENT,
    fecha_inicio date,
    fecha_fin date,
    camara_id INT NOT NULL,
    tipo VARCHAR(15) NOT NULL,
    PRIMARY KEY (id),
    FOREIGN KEY (camara_id) REFERENCES camaras(id),
    FOREIGN KEY (tipo) REFERENCES tipos_de_sesion(tipo)
);

# Control
CREATE TABLE controles (
    id INT NOT NULL AUTO_INCREMENT,
    empleado_id INT,
    PRIMARY KEY (id),
    FOREIGN KEY (empleado_id) REFERENCES empleados(id)
);

# Proyecto de ley
CREATE TABLE proyectos_de_ley (
      id INT NOT NULL AUTO_INCREMENT,
      fecha_inicio DATE,
      camara_id INT NOT NULL,
      titulo VARCHAR(127),
      control_id INT, # aclara que se controla cada proyecto como mucho una vez
      aprobado_diputados BOOLEAN DEFAULT 0,
      aprobado_senadores BOOLEAN DEFAULT 0,
      PRIMARY KEY (id),
      FOREIGN KEY (camara_id) REFERENCES camaras(id),
      FOREIGN KEY (control_id) REFERENCES controles(id)
);

# Ley
CREATE TABLE leyes (
    id INT NOT NULL AUTO_INCREMENT,
    año INT,
    sesion_id INT NOT NULL,
    proyecto_de_ley_id INT NOT NULL,
    PRIMARY KEY (id),
    FOREIGN KEY (sesion_id) REFERENCES sesiones(id),
    FOREIGN KEY (proyecto_de_ley_id) REFERENCES proyectos_de_ley(id)
);

# TipoVoto
CREATE TABLE tipos_de_voto (
    tipo VARCHAR(15) NOT NULL,
    PRIMARY KEY (tipo)	
);

# Ternaria entre ciudadano, sesion y proyecto de ley
CREATE TABLE votos (
    ciudadano_id INT NOT NULL,
    sesion_id INT NOT NULL,
    proyecto_de_ley_id INT NOT NULL,
    tipo_de_voto VARCHAR(15) NOT NULL,
    PRIMARY KEY (ciudadano_id, proyecto_de_ley_id),
    FOREIGN KEY (ciudadano_id) REFERENCES ciudadanos(id),
    FOREIGN KEY (sesion_id) REFERENCES sesiones(id),
    FOREIGN KEY (proyecto_de_ley_id) REFERENCES proyectos_de_ley(id),
    FOREIGN KEY (tipo_de_voto) REFERENCES tipos_de_voto(tipo)
);

# Comision
CREATE TABLE comisiones (
    id INT NOT NULL AUTO_INCREMENT,
    camara_diputados_id INT NOT NULL,
    nombre VARCHAR(127),
    presidente_id INT,
    PRIMARY KEY (id),
    FOREIGN KEY (camara_diputados_id) REFERENCES camaras_diputados(id),
    FOREIGN KEY (presidente_id) REFERENCES diputados(id)
);

# Diputados con comisiones (pertenece)
CREATE TABLE integrantes_comisiones (
    diputado_id INT NOT NULL,
    comision_id INT NOT NULL,
    PRIMARY KEY (diputado_id, comision_id),
    FOREIGN KEY (diputado_id) REFERENCES diputados(id),
    FOREIGN KEY (comision_id) REFERENCES comisiones(id) 
);

# Proyectos de ley con comisiones
CREATE TABLE proyectos_de_ley_comisiones (
    proyecto_de_ley_id INT NOT NULL,
    comision_id INT NOT NULL,
    informante_id INT,
    PRIMARY KEY (proyecto_de_ley_id, comision_id),
    FOREIGN KEY (proyecto_de_ley_id) REFERENCES proyectos_de_ley(id),
    FOREIGN KEY (comision_id) REFERENCES comisiones(id),
    FOREIGN KEY (informante_id) REFERENCES diputados(id)  
);

# Relacion ciudadanos sesiones (asistencias) 
CREATE TABLE  asistencias(
    ciudadano_id INT NOT NULL,
    sesion_id INT NOT NULL,
    PRIMARY KEY (ciudadano_id, sesion_id),
    FOREIGN KEY (ciudadano_id) REFERENCES ciudadanos(id),
    FOREIGN KEY (sesion_id) REFERENCES sesiones(id) 
);

# Bienes economicos
CREATE TABLE bienes_economicos (
  id INT NOT NULL AUTO_INCREMENT,
  ciudadano_id INT NOT NULL,
  valor FLOAT NOT NULL,
  año INT NOT NULL,
  detalles VARCHAR(255),
  PRIMARY KEY (id),
  FOREIGN KEY (ciudadano_id) REFERENCES ciudadanos(id)
);

CREATE VIEW declaraciones_juradas AS
SELECT ciudadano_id, año, COALESCE(SUM(valor),0) patrimonio
FROM bienes_economicos
GROUP BY año, ciudadano_id;

\end{verbatim}

\subsection{Funcionalidades implementadas}

\textbf{Los Diputados que sólo votaron positivo o ausente en los proyectos de ley que pertenecen a la comisión que integran.}

\begin{verbatim}
CREATE VIEW voto_proyectos_que_pertenece AS
SELECT i.diputado_id, p.proyecto_de_ley_id, v.tipo_de_voto
FROM integrantes_comisiones i
JOIN proyectos_de_ley_comisiones p ON i.comision_id = p.comision_id
JOIN votos v ON v.ciudadano_id = i.diputado_id AND v.proyecto_de_ley_id = p.proyecto_de_ley_id;


SELECT DISTINCT c.*
FROM ciudadanos c
INNER JOIN voto_proyectos_que_pertenece p ON c.id = p.diputado_id
WHERE c.id NOT IN (
               SELECT DISTINCT diputado_id
               FROM voto_proyectos_que_pertenece v
               WHERE v.tipo_de_voto NOT IN ('afirmativo', 'ausente')
);

\end{verbatim}

\textbf{Cantidad de Leyes promulgadas en cada sesión en los últimos 3 años.}

\begin{verbatim}
SELECT sesiones.id, count(leyes.id) cantidad
FROM sesiones 
LEFT JOIN leyes ON sesiones.id = leyes.sesion_id
WHERE year(fecha_inicio) >= year(now()) - 2
GROUP BY sesiones.id;

\end{verbatim}

\newpage
\textbf{Los diez Legisladores con mayor incremento porcentual desde que se
iniciaron en el cargo.}

\begin{verbatim}
SELECT c.id, c.nombre, c.apellido,
      IF (dji.patrimonio = 0, 100, (djf.patrimonio / dji.patrimonio - 1) * 100) incremento_porcentual
FROM declaraciones_juradas dji
INNER JOIN ciudadanos c ON c.id = dji.ciudadano_id
INNER JOIN declaraciones_juradas djf ON dji.ciudadano_id = djf.ciudadano_id
WHERE djf.año = (
  SELECT MAX(año)
  FROM declaraciones_juradas dj
  WHERE dj.ciudadano_id = c.id)
AND dji.año = (
  SELECT MIN(año)
  FROM declaraciones_juradas dj
  WHERE dj.ciudadano_id = c.id)
ORDER BY incremento_porcentual DESC
LIMIT 10;

\end{verbatim}

\textbf{Resolución de alguna de las implicancias del problema utilizando
triggers.}

\begin{verbatim}
DELIMITER $$
 
CREATE TRIGGER chequear_edad_diputados
     BEFORE INSERT ON diputados FOR EACH ROW
     BEGIN
          IF (year(now()) - (SELECT year(fecha_nacimiento) FROM ciudadanos WHERE id = NEW.id)) < 25 THEN 
     		SIGNAL SQLSTATE '45000'
                    SET MESSAGE_TEXT = 'No se puede ser diputado con esa edad';
     	END IF;
     END;
$$

DELIMITER $$ 
 
CREATE TRIGGER chequeo_representaciones
     BEFORE INSERT ON representaciones FOR EACH ROW
     BEGIN
        IF NEW.camara_id = 'senadores' AND 
           (year(now()) - (SELECT year(fecha_nacimiento) FROM ciudadanos WHERE id = NEW.ciudadano_id)) < 30 THEN 
     		SIGNAL SQLSTATE '45000'
                    SET MESSAGE_TEXT = 'No se puede ser senador con esa edad';
     	END IF;
     	
     	IF EXISTS (SELECT id 
				   FROM representaciones 
				   INNER JOIN camaras ON representaciones.camara_id = camaras.id 
                   WHERE ciudadano_id = NEW.ciudadano_id AND 
                   año = (SELECT año FROM camaras WHERE id = NEW.camara_id) 
                   ) THEN 
     		SIGNAL SQLSTATE '45000'
                    SET MESSAGE_TEXT = 'Solo puede estar en una camara para un mismo año';
     	END IF;
     	
     END;
$$

DELIMITER $$

CREATE TRIGGER chequear_edad_ciudadanos
	BEFORE INSERT ON ciudadanos FOR EACH ROW
	BEGIN
		IF (year(now()) - year(NEW.fecha_nacimiento)) < 18 THEN 
			SIGNAL SQLSTATE '45000'
                    SET MESSAGE_TEXT = 'No puede ingresar un ciudadano menor a 18 años';
		END IF;
		
	END;
$$

DELIMITER $$

CREATE TRIGGER no_agregar_asistencia_si_voto_ausente
     BEFORE INSERT ON asistencias FOR EACH ROW
     BEGIN
          IF (SELECT count(*) 
          		FROM votos 	
          		WHERE ciudadano_id = NEW.ciudadano_id AND
          		sesion_id = NEW.sesion_id AND
          		tipo_de_voto = 'ausente' 
              ) > 0 THEN 
     		SIGNAL SQLSTATE '45000'
                    SET MESSAGE_TEXT = 'Hay un voto ausente para ese ciudadano en esa sesion';
     	END IF;
     END;
$$

DELIMITER $$

CREATE TRIGGER no_agregar_voto_ausente_si_asistencia
     BEFORE INSERT ON votos FOR EACH ROW
     BEGIN
          IF (NEW.tipo_de_voto = 'ausente' AND
             (SELECT count(*) 
          		FROM asistencias 	
          		WHERE ciudadano_id = NEW.ciudadano_id AND
          		sesion_id = NEW.sesion_id
              ) > 0) THEN 
     		SIGNAL SQLSTATE '45000'
                    SET MESSAGE_TEXT = 'No se puede agregar un voto ausente porque hay datos que verifican que el legislador estuvo presente en esa sesion';
     	END IF;
     END;
$$


DELIMITER $$ 
 
CREATE TRIGGER chequear_ley_aprobada_en_ambas_camaras
     BEFORE INSERT ON leyes FOR EACH ROW
     BEGIN
        IF (EXISTS
        		(SELECT id FROM proyectos_de_ley 
        				   WHERE id = NEW.proyecto_de_ley_id AND
        				   (aprobado_diputados = 0 OR
        				   aprobado_senadores = 0)
        		)
        	)THEN 
     		SIGNAL SQLSTATE '45000'
                    SET MESSAGE_TEXT = 'No esta aprobado el proyecto en ambas camaras';
     	END IF;
     	
     END;
$$
\end{verbatim}
