c)Modelo de Entidad Relación y Modelo Relacional derivado, utilizados para implementar la solución.
d)Detalle de los supuestos asumidos para la resolución del problema.


\section{Modelo de Entidad Relación}

(Guardamos historias de los partidos y bloque)
(el presidente es elegido por sus pares)
(tipos de sesiones?)
(una comisión tiene un presidente?)
(ciudadano -> legislador. Para poder def el empleado..)



Las restricciones que no pueden ser modeladas en el diagrama son expresadas a continuación utilizando el lenguaje natural:
\begin{itemize}
	\item Por cada 33000 habitantes de una Provincia, deberá haber un Diputado que la represente en la Cámara de Diputados. 
	\item Un Legislador puede ser Diputado si tiene más de 25 años.
	\item Un Legislador puede ser Senador si tiene más de 30 años.
	\item El Vicepresidente de la Nación no es Senador.
	\item La Fecha de Inicio de una Sesión debe estar comprendida entre el 1 de Marzo hasta el 30 de noviembre del mismo año.	
	\item La Fecha de Fin de una Sesión debe estar comprendida entre el 1 de Marzo hasta el 30 de noviembre del mismo año.
	\item La Fecha de Inicio de una Sesión debe ser menor a la Fecha de Fin de la misma. 
	\item Una Ley solo puede existir si su correspondiente proyecto fue aprobado por la mayoría de ambas Cámaras (de Diputados y de Senadores).
	\item En la Cámara de Diputados debe haber, en todo momento, 45 Comisiones Parlamentarias.
	\item Un Legislador solo puede votar, o el mismo ser considerado válido, si tiene menos de 15 ausencias anuales.  

\end{itemize}

\section{Modelo Relacional}

