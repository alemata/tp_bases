\section{Ejemplos de la vida cotidiana}

\subsection{Distribución Normal}
\begin{itemize}
\item Ejemplo 1: Un ejemplo de la vida real cuya información presenta una distribución normal, puede ser el de los datos metereológicos correspondientes a temperaturas, lluvias, etc. 

\item Ejemplo 2: Otro ejemplo es el de la vida media de un producto electrónico. Al fabricarlo, se espera un tiempo de vida útil, pero el mismo puede ocurrir ser menor o mayor según el uso que se le de, mientras que el esperado en general se cumplirá. 

\item Dataset:

\end{itemize}

\subsection{Distribución Uniforme}
\begin{itemize}
\item Ejemplo 1: En la vida real se puede observar que las edades de las personas de cierta ciudad presentan una distrubución informe. 

\item Ejemplo 2: Otro caso a tener en cuenta, que respeta una distribución uniforme es el de las frecuencias en las que un tren en Buenos Aires arriba a una estación. Se tiene en cuenta únicamente el horario en el que el tren funciona. 

\item Dataset:

\end{itemize}

